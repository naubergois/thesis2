%Seção de resumo

\begin{resumo} 
 
Alguns sistemas de software devem responder a milhares ou milhões de pedidos simultâneos.
Esses sistemas devem ser devidamente testados para garantir que eles possam funcionar corretamente sob a carga esperada. A degradação do desempenho e conseqüentes falhas do sistema geralmente ocorrem em condições de estresse. O teste de estresse sujeita o programa a cargas pesadas. Os testes de estresse diferem de outros tipos de testes em que o sistema é executado em seus pontos de interrupção, forçando o aplicativo ou a infra-estrutura de suporte a falhar. A busca do tempo de execução mais longo é vista como um problema de otimização descontínuo, não-linear, com o domínio de entrada do sistema em teste como espaço de busca. Neste contexto, os testes baseados em pesquisa (\textit{search-based tests}) são vistos como uma abordagem promissora para verificar as restrições de tempo. O teste de software baseado em pesquisa é a aplicação de técnicas de pesquisa metaheurística para gerar testes de software. O critério de adequação do teste é transformado em uma função de fitness e um conjunto de soluções no espaço de busca é avaliado em relação à função de fitness usando uma metaheurística. O teste de estresse baseado em pesquisa envolve encontrar os tempos de execução melhores e piores para verificar se as restrições de tempo são cumpridas. Os acordos de nível de serviço (SLA) são documentos que especificam garantias de desempenho realistas, bem como penalidades por incumprimento. Os SLAs são feitos entre provedores e clientes que incluem qualidade do serviço, capacidade de recursos, escalabilidade, obrigações e conseqüências em caso de violação. Satisfazer o SLA é de grande importância e um problema desafiador. A principal motivação desta tese é encontrar o tempo de resposta adequado dos SLAs usando teste de estresse. Esta tese aborda três abordagens em testes de estresse baseados em pesquisa. Primeiro, a metaheurística híbrida usa Tabu Search, Simulated Annealing e Algoritmos Genéticos de forma colaborativa. Em segundo lugar, uma abordagem chamada HybridQ usa uma técnica de aprendizado de reforço para otimizar a escolha de soluções vizinhas para explorar, reduzindo o tempo necessário para obter os cenários com o tempo de resposta mais longo na aplicação. As melhores soluções encontradas pelo HybridQ foram em média 5,98 \% melhores que alcançados pela abordagem híbrida sem Q-learning. Em terceiro lugar, a tese investiga o uso dos algoritmos multi-objetivos NSGA-II, SPEA2, PAES e MOEA/D. A metaheurística MOEA/D obteve o melhor valor de hipervolume quando comparada com outras abordagens. A abordagem colaborativa usando MOEA/D e HybridQ melhora os valores de hipervolume obtidos e encontrou \textit{workloads} mais relevantes do que as experiências anteriores. Uma ferramenta chamada IAdapter, um plugin JMeter para realizar testes de esforço baseados em pesquisa, foi desenvolvida e usada para realizar todas as experiências.

\textbf{Palavras-chave}: Search-based Testing, Stress Testing, Multi-objective metaheuristics, Hybrid metaheuristics, Reinforcement Learning.

\end{resumo}

