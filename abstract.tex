%Resumo em inglês
\begin{abstract}

Some software systems must respond to thousands or millions of concurrent requests. These systems must be properly tested to ensure that they can function correctly under the expected load. Performance degradation and consequent system failures usually arise in stressed conditions. Stress testing subjects the program to heavy loads. In this context, search-based testing is seen as a promising approach to verify timing constraints.
Service Level Agreements (SLAs)  are documents that specify realistic performance guarantees as well as
penalties for non-compliance.   SLAs are made between providers and customers that include service quality, resources
capability, scalability, obligations and consequences in case of
violations. Satisfying SLA is very important and a challenging
issue. The main motivation of this thesis was helps to define the corrected response time of SLAs using Stress Testing. This thesis addresses the use of hybrid and multi-objective metaheuristics in conjunction with reinforcement learning techniques in search-based stress tests. A tool named IAdapter (www.iadapter.org, github.com/naubergois/newiadapter), a JMeter plugin for performing search-based stress tests, was developed.  Several experiments was conducted to validate the proposed approach. The thesis proposes a Hybrid, HybridQ and a multi-objective apporach. Hybrid  algorithm uses Tabu Search, Simulated Annealing and Genetic Algorithms in a collaborative manner. HybridQ uses a reinforcement learning technique to optimize the choice of neighboring solutions to explore, reducing the time needed to obtain the scenarios with the longest response time in the application. Finally, thesis investigates the use of multi-objective algorithm in search-based stress testing.



\textbf{Keywords}: Search-based Testing, Stress Testing, Multi-objective metaheuristics, Hybrid metaheuristics, Reinforcement Learning.
 
\end{abstract}
 
