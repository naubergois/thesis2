%Resumo em inglês
\begin{abstract}

Some software systems must respond to thousands or millions of concurrent requests.
These systems must be properly tested to ensure that they can function correctly under the expected load. Performance degradation and consequent system failures usually arise in stressed conditions. Stress testing subjects the program to heavy loads. Stress tests differ from other kinds of testing in that the system is executed on its breakpoints, forcing the application or the supporting infrastructure to fail. The search for the longest execution time is seen as a discontinuous, nonlinear, optimization problem, with the input domain of the system under test as a search space. In this context, search-based testing is viewed as a promising approach to verify timing constraints. Search-based software testing is the application of metaheuristic search techniques to generate software tests. The test adequacy criterion is transformed into a fitness function and a set of solutions in the search space is evaluated with respect to the fitness function using a metaheuristic. Search-based stress testing involves finding the best- and worst-case execution times to ascertain whether timing constraints are fulfilled. Service Level Agreements (SLAs) are documents that specify realistic performance guarantees as well as penalties for non-compliance. SLAs are made between providers and customers that include service quality, resources capability, scalability, obligations, and consequences in case of violations. Satisfying SLA is of great importance and a challenging issue. The main motivation of this thesis is to find the adequate response time of SLAs using Stress Testing. This thesis addresses three approaches in search-based stress tests. First, Hybrid metaheuristic uses Tabu Search, Simulated Annealing, and Genetic Algorithms in a collaborative manner. Second, an approach called HybridQ uses a reinforcement learning technique to optimize the choice of neighboring solutions to explore, reducing the time needed to obtain the scenarios with the longest response time in the application. The best solutions found by HybridQ were on average 5.98\% better that achieved by the Hybrid approach without Q-learning. Third, the thesis investigates the use of the multi-objective NSGA-II, SPEA2, PAES and MOEA/D algorithms. MOEA/D metaheuristics obtained the best hypervolume value when compared with other approaches. The collaborative approach using MOEA/D and HybridQ improves the hypervolume values obtained and found more relevant workloads than the previous experiments. A tool named IAdapter, a JMeter plugin for performing search-based stress tests, was developed and used to conduct all the experiments.

\textbf{Keywords}: Search-based Testing, Stress Testing, Multi-objective metaheuristics, Hybrid metaheuristics, Reinforcement Learning.
 
\end{abstract}
 
