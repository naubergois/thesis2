%Seção de agradecimentos
\chapter*{Acknowledgments} 

%\begin{trivlist} \itemsep 2ex \normalsize


It has been an exciting and instructive study period in the University of Fortaleza and I
feel privileged to have had the opportunity to carry out this study as demonstration of
knowledge gained during the period. With these acknowledgments, it would be impossible to remember those who in one way or another, directly or indirectly, have played a role in the realization of this research project. 

First of all, thank you God for the blessing and guiding me during all this project. I would like to dedicate this thesis to three of the most important people in my life
– my wife, my mother and my father – whose unconditional love and support gave me the
strength to follow my dream of becoming a scientist. I express all my gratitude to my beloved wife Najara, who endured five long years by my side without ever questioning my choice of becoming a doctor degree student. Thank you to my parents, Gois and Zirlanda,  for the example of life and for encouraging me as a child to enjoy science.

I would like to thank my very dedicated advisors Dr. Pedro Porfírio and André Coelho, for his guidance and support throughout this research project, which has been an excellent
learning experience for me. This thesis is not solely the product of my work but also the hard work and perseverance of my advisors, providing  help to keep me going even in the most difficult of times to get the project done. I could only hope that more students have the opportunity to work and learn from both professors like I did.

I also would like to thank the following people:

My thesis committee – Dr. Pedro de Alcantara dos Santos Neto, Dr. João Paulo Pordeus Gomes, Ph.D. Americo Tadeu Falcone Sampaio and Docteur Napoleão Vieira Nepomuceno – who made themselves available when I needed them to review previous drafts of this thesis and for their valuable suggestions that helped me to bring this work to its completion.

My family who loves me and are always present for good or bad: my mother in law Aroldina, my brothers Nilton and Nelio, my aunt Zena and my grandmother Elisabete.

Adriano Bessa, Victor and all professors from the post-graduate program for their direct and indirect contributions.

My colleagues from UNIFOR who supported me during this project: Adriana, Marum Simão, Celso Medeiros and José Renato. 

My company, SERPRO—Brazilian Federal Datacenter, for the (partial) financial support. My colleagues  Flavio Cysne, Fred Viana and Suderland Guimarães to help me with one of the experiments applied in this thesis. Sergio Gomes and Carlos Henrique (Managers of The Fortaleza Team) for giving me the time necessary to develop my activities.


