%Resumo em inglês
\begin{abstract}

Some software systems must respond to thousands or millions of concurrent requests. These systems must be properly tested to ensure that they can function correctly under the expected load. Performance degradation and consequent system failures usually arise in stressed conditions. Stress testing subjects the program to heavy loads. Stress tests differ from other kinds of testing in that the system is executed on its break-points, forcing the application or the supporting infrastructure to fail. The search for the longest execution time is seen as a discontinuous, nonlinear, optimization problem, with the input domain of the system under test as a search space. In this context, search-based testing is viewed as a promising approach to verify timing constraints. Search-based software testing is the application of metaheuristic search techniques to generate software tests. The test adequacy criterion is transformed into a fitness function and a set of solutions in the search space are evaluated with respect to the fitness function using a metaheuristic. Search-based stress testing involves finding the best- and worst-case execution times to ascertain whether timing constraints are fulfilled. Service Level Agreements (SLAs)  are documents that specify realistic performance guarantees as well as
penalties for non-compliance.   SLAs are made between providers and customers that include service quality, resources
capability, scalability, obligations and consequences in case of
violations. Satisfying SLA is of great importance and a challenging
issue. The main motivation of this thesis is helps to define the  response time of SLAs using Stress Testing. This thesis addresses the use of hybrid and multi-objective metaheuristics in conjunction with reinforcement learning techniques in search-based stress tests. A tool named IAdapter, a JMeter plugin for performing search-based stress tests, was developed.  The thesis  uses Tabu Search, Simulated Annealing and Genetic Algorithms in a collaborative manner. HybridQ uses a reinforcement learning technique to optimize the choice of neighboring solutions to explore, reducing the time needed to obtain the scenarios with the longest response time in the application.   Finally, thesis investigates the use of multi-objective algorithm in search-based stress testing. Several experiments were conducted to validate the proposed approach. The present research compared the proposed approach with other methods based on the use of single or multiobjective metaheuristics. The experiments showed  a significant improvement in results with the Hybrid Metaheuristic approach. The best solutions found by HybridQ were on average 5.98\% better than that achieved by our previous approach without q-learning. The use of multiobjective metaheuristics has made it possible to delimit a frontier where the response times are below the previously established service level. MOEA/D metaheuristics obtained the best hypervolume value when compared with other approaches. However, we have verified from the experiments that some of the results found by the HybridQ algorithm can contribute to the use of MOEA/D.  The collaborative approach using MOEA/D and HybridQ improves the hypervolume values obtained and found more relevant workloads than the previous experiments. Also as a typical search strategy, it is difficult to ensure that the execution times generated in the experiments represent global optimum. More experimentation is also required to determine the most appropriate and robust parameters and verify changes related to the Pareto frontier in different executions. 



\textbf{Keywords}: Search-based Testing, Stress Testing, Multi-objective metaheuristics, Hybrid metaheuristics, Reinforcement Learning.
 
\end{abstract}
 
